\def\RR{{\mathbb{R}}}
\def\EE{{\mathbb{E}}}
\def\RRY{\RR^{|\cY|}}
\def\y{\bm{y}}
\def\triangleY{\triangle^{|\cY|}}
\def\sizeY{{|\cY|}}
\def\cC{{\mathcal{C}}}
\def\cD{{\mathcal{D}}}
\def\cX{{\mathcal{X}}}
\def\cY{{\mathcal{Y}}}

\fancychapter{Proof of Proposition~\ref{prop:grad_alpha}}\label{app:fenchelyoung}

\cleardoublepage
\setstretch{1.5}

\section*{Jacobian of {\boldmath $\alpha$}-\entmaxtext \wrt the shape parameter
      {\boldmath $\alpha$}}
\label{app:alpha_grad}

Recall that the \entmaxtext transformation is defined as:
\begin{equation}\label{eq:entmax_form_supp}
    \aentmax(\bm{z}) \coloneqq
    \argmax_{\p \in \simplex^d} \bm{p}^\top\bm{z} + \HHt_{\alpha}(\bm{p}),
\end{equation}
where $\alpha \geq 1$ and $\HHt_{\alpha}$ is the Tsallis entropy,
\begin{equation}%
    \HHt_{\alpha}(\bm{p})\!\coloneqq\!
    %(\alpha-1)^{-1} \left(1 - \sum_j p_j^{\alpha}\right)
    %= k(\alpha-1)^{-1}(1 - \|\p\|_{\alpha}^{\alpha}
    \begin{cases}
        \frac{1}{\alpha(\alpha-1)}\sum_j\!\left(p_j - p_j^\alpha\right)\!, &
        \!\!\!\alpha \ne 1,                                                  \\
        \HHs(\bm{p}),                                                      &
        \!\!\!\alpha = 1,
    \end{cases}
\end{equation}
and $\HHs(\bm{p}):= -\sum_j p_j \log p_j$ is the Shannon entropy.

In this section, we derive the Jacobian of $\entmax$ with respect to the scalar parameter $\alpha$.

\section*{General case of {\boldmath $\alpha>1$}}

From the KKT conditions associated with the optimization problem in
Eq.~\ref{eq:entmax_form_supp}, we have that the solution $\bm{p}^{\star}$ has the following form, coordinate-wise:
\begin{equation}\label{eq:p_kkt}
    p_i^{\star} = [(\alpha-1)(z_i - \tau^{\star})]_+^{1/(\alpha-1)},
\end{equation}
where $\tau^{\star}$ is a scalar Lagrange multiplier that ensures that
$\bm{p}^{\star}$ normalizes to 1, \ie, it is defined implicitly by the condition:
\begin{equation}\label{eq:tau_condition}
    \sum_i [(\alpha-1)(z_i - \tau^{\star})]_+^{1/(\alpha-1)} = 1.
\end{equation}
For general values of $\alpha$, Eq.~\ref{eq:tau_condition} lacks a closed form solution. This makes the computation of the
Jacobian
\begin{equation}
    \frac{\partial \aentmax(\bm{z})}{\partial \alpha}
\end{equation}
non-trivial. Fortunately, we can use the technique of implicit differentiation
to obtain this Jacobian.

The Jacobian exists almost everywhere, and the expressions we derive
expressions yield a generalized Jacobian~\citep{clarke_book} at any
non-differentiable points that may occur for certain ($\alpha$, $\x$)
pairs. We begin by noting that $\frac{\partial p_i^{\star}}{\partial
    \alpha} = 0$ if $p_i^{\star} = 0$, because increasing $\alpha$ keeps
sparse coordinates sparse.\footnote{This follows from the margin
    property of $\HHt_\alpha$~\citep{blondel2019learning}.}
Therefore we need to worry only
about coordinates that are in the support of $\bm{p}^\star$. We will assume
hereafter that the $i$\textsuperscript{th} coordinate of $\bm{p}^\star$ is non-zero.
We have:
\begin{eqnarray}\label{eq:gradient_alpha_01}
    \frac{\partial p_i^{\star}}{\partial \alpha} &=& \frac{\partial}{\partial \alpha} [(\alpha-1)(z_i - \tau^{\star})]^{\frac{1}{\alpha-1}}\nonumber\\
    &=& \frac{\partial}{\partial \alpha} \exp \left[\frac{1}{\alpha-1} \log [(\alpha-1)(z_i - \tau^{\star})]\right]\nonumber\\
    &=& p_i^{\star} \frac{\partial}{\partial \alpha} \left[\frac{1}{\alpha-1} \log [(\alpha-1)(z_i - \tau^{\star})]\right]\nonumber\\
    &=& \frac{p_i^{\star}}{(\alpha-1)^2} \left[\frac{\frac{\partial}{\partial \alpha} [(\alpha-1)(z_i - \tau^{\star})]}{z_i - \tau^{\star}} - \log[(\alpha-1)(z_i - \tau^{\star})] \right]\nonumber\\
    &=& \frac{p_i^{\star}}{(\alpha-1)^2} \left[\frac{z_i - \tau^{\star} - (\alpha-1)\frac{\partial \tau^{\star}}{\partial \alpha} }{z_i - \tau^{\star}} - \log[(\alpha-1)(z_i - \tau^{\star})] \right]\nonumber\\
    &=& \frac{p_i^{\star}}{(\alpha-1)^2} \left[1 - \frac{\alpha-1}{z_i - \tau^{\star}}\frac{\partial \tau^{\star}}{\partial \alpha} - \log[(\alpha-1)(z_i - \tau^{\star})] \right].
\end{eqnarray}
We can see that this Jacobian depends on $\frac{\partial \tau^{\star}}{\partial \alpha}$, which we now compute using implicit differentiation.

Let $\mathcal{S} = \{i: \pp^\star_i > 0 \}$).
By differentiating both sides of Eq.~\ref{eq:tau_condition}, re-using some of
the steps in Eq.~\ref{eq:gradient_alpha_01}, and recalling Eq.~\ref{eq:p_kkt},
we get
\begin{eqnarray}\label{eq:gradient_tau_implicit}
    0 &=& \sum_{i \in \mathcal{S}} \frac{\partial}{\partial \alpha} [(\alpha-1)(z_i - \tau^{\star})]^{1/(\alpha-1)}\nonumber\\
    &=& \sum_{i \in \mathcal{S}} \frac{p_i^{\star}}{(\alpha-1)^2} \left[1 - \frac{\alpha-1}{z_i - \tau^{\star}}\frac{\partial \tau^{\star}}{\partial \alpha} - \log[(\alpha-1)(z_i - \tau^{\star})] \right]\nonumber\\
    &=&  \frac{1}{(\alpha-1)^2} - \frac{\partial \tau^{\star}}{\partial \alpha} \sum_{i \in \mathcal{S}} \frac{p_i^{\star}}{(\alpha - 1)(z_i - \tau^{\star})} - \sum_{i \in \mathcal{S}} \frac{p_i^{\star}}{(\alpha-1)^2} \log[(\alpha-1)(z_i - \tau^{\star})] \nonumber\\
    &=&  \frac{1}{(\alpha-1)^2} - \frac{\partial \tau^{\star}}{\partial \alpha} \sum_{i} (p_i^{\star})^{2-\alpha} - \sum_{i} \frac{p_i^{\star}}{\alpha-1} \log p_i^{\star}\nonumber\\
    &=&  \frac{1}{(\alpha-1)^2} - \frac{\partial \tau^{\star}}{\partial \alpha} \sum_{i} (p_i^{\star})^{2-\alpha} + \frac{\HHs(\bm{p}^*)}{\alpha-1},
\end{eqnarray}
from which we obtain:
\begin{eqnarray}\label{eq:gradient_tau}
    \frac{\partial \tau^{\star}}{\partial \alpha} &=& \frac{\frac{1}{(\alpha-1)^2} + \frac{\HHs(\bm{p}^{\star})}{\alpha-1}}{\sum_i (p_i^{\star})^{2-\alpha}}.
\end{eqnarray}

Finally, plugging Eq.~\ref{eq:gradient_tau} into Eq.~\ref{eq:gradient_alpha_01}, we get:
\begin{eqnarray}\label{eq:gradient_alpha}
    \frac{\partial p_i^{\star}}{\partial \alpha} &=&  \frac{p_i^{\star}}{(\alpha-1)^2} \left[1 - \frac{1}{(p_i^{\star})^{\alpha-1}}\frac{\partial \tau^{\star}}{\partial \alpha} - (\alpha-1)\log p_i^{\star} \right]\nonumber\\
    &=&  \frac{p_i^{\star}}{(\alpha-1)^2} \left[1 - \frac{1}{(p_i^{\star})^{\alpha-1}}\frac{\frac{1}{(\alpha-1)^2} + \frac{\HHs(\bm{p}^{\star})}{\alpha-1}}{\sum_i (p_i^{\star})^{2-\alpha}} - (\alpha-1)\log p_i^{\star} \right]\nonumber\\
    &=& \frac{p_i^{\star} - \tilde{p}_i(\alpha)}{(\alpha-1)^2} - \frac{p_i^{\star}\log p_i^{\star} + \tilde{p}_i(\alpha)\HHs(\bm{p}^{\star})}{\alpha-1},
\end{eqnarray}
where we denote by
\begin{equation}
    \tilde{p}_i(\alpha) = \frac{(p_i^{\star})^{2-\alpha}}{\sum_j (p_j^{\star})^{2-\alpha}}.
\end{equation}
The distribution $\tilde{\bm{p}}(\alpha)$ can be interpreted as a ``skewed''
distribution obtained from $\bm{p}^{\star}$, which appears in the Jacobian of
$\aentmax(\x)$ \wrt $\x$ as well~\cite{entmax}.


\section*{Solving the indetermination for {\boldmath $\alpha=1$}}

We can write Eq.~\ref{eq:gradient_alpha} as
\begin{eqnarray}\label{eq:gradient_alpha_fraction}
    \frac{\partial p_i^{\star}}{\partial \alpha} &=&
    \frac{p_i^{\star} - \tilde{p}_i(\alpha) - (\alpha-1)(p_i^{\star}\log p_i^{\star} + \tilde{p}_i(\alpha)\HHs(\bm{p}^{\star}))}{(\alpha-1)^2}.
\end{eqnarray}
When $\alpha \rightarrow 1^+$, we have $\tilde{\bm{p}}(\alpha) \rightarrow \bm{p}^{\star}$, which leads to a $\frac{0}{0}$ indetermination.

To solve this indetermination, we will need to apply L'H\^opital's rule twice.
Let us first compute the derivative of $\tilde{p}_i(\alpha)$ with respect to $\alpha$. We have
\begin{equation}
    \frac{\partial}{\partial \alpha} (p_i^\star)^{2-\alpha} = -(p_i^{\star})^{2-\alpha} \log p_i^{\star},
\end{equation}
therefore
\begin{eqnarray}
    \frac{\partial}{\partial \alpha} \tilde{p}_i(\alpha) &=& \frac{\partial}{\partial \alpha} \frac{(p_i^\star)^{2-\alpha}}{\sum_j (p_j^\star)^{2-\alpha}}\nonumber\\
    &=& \frac{-(p_i^{\star})^{2-\alpha} \log p_i^{\star} \sum_j (p_j^\star)^{2-\alpha} + (p_i^{\star})^{2-\alpha} \sum_j (p_j^{\star})^{2-\alpha} \log p_j^{\star}}{\left( \sum_j (p_j^\star)^{2-\alpha} \right)^2}\nonumber\\
    &=& -\tilde{p}_i(\alpha)\log p_i^{\star} + \tilde{p}_i(\alpha) \sum_j \tilde{p}_j(\alpha) \log p_j^{\star}.
\end{eqnarray}
Differentiating the numerator and denominator in Eq.~\ref{eq:gradient_alpha_fraction}, we get:
\begin{eqnarray}\label{eq:gradient_alpha_shannon_01}
    \frac{\partial p_i^{\star}}{\partial \alpha} &=&
    \lim_{\alpha \rightarrow 1^+} \frac{(1 + (\alpha-1)\HHs(\bm{p}^{\star})) \tilde{p}_i(\alpha) (\log p_i^{\star} - \sum_j \tilde{p}_j(\alpha) \log p_j^{\star}) - p_i^{\star}\log p_i^{\star} - \tilde{p}_i(\alpha) \HHs(\bm{p}^{\star})}{2(\alpha-1)} \nonumber\\
    &=& A + B,
\end{eqnarray}
with
\begin{eqnarray}\label{eq:A_shannon}
    A &=& \lim_{\alpha \rightarrow 1^+} \frac{\HHs(\bm{p}^{\star}) \tilde{p}_i(\alpha) (\log p_i^{\star} - \sum_j \tilde{p}_j(\alpha) \log p_j^{\star}) \HHs(\bm{p}^{\star})}{2}\nonumber\\
    &=& \frac{\HHs(\bm{p}^{\star}) p_i^{\star}\log p_i^{\star} + p_i^{\star} (\HHs(\bm{p}^{\star}))^2}{2},
\end{eqnarray}
and
\begin{equation}\label{eq:B_shannon}
    B = \lim_{\alpha \rightarrow 1^+} \frac{\tilde{p}_i(\alpha) (\log p_i^{\star} - \sum_j \tilde{p}_j(\alpha) \log p_j^{\star}) - p_i^{\star}\log p_i^{\star} - \tilde{p}_i(\alpha) \HHs(\bm{p}^{\star})}{2(\alpha-1)}.
\end{equation}
When $\alpha\rightarrow 1^+$, $B$ becomes again a $\frac{0}{0}$ indetermination, which we can solve by applying again L'H\^opital's rule. Differentiating the numerator and denominator in Eq.~\ref{eq:B_shannon}:
\begin{eqnarray}\label{eq:B_shannon_02}
    B &=& \frac{1}{2}\lim_{\alpha \rightarrow 1^+} \left\{ \tilde{p}_i(\alpha) \log p_i^{\star} \left(\sum_j \tilde{p}_j(\alpha) \log p_j^{\star} - \log p_i^{\star}\right) \right. \nonumber\\
    && - \tilde{p}_i(\alpha) \left(\sum_j \tilde{p}_j(\alpha) \log p_j^{\star} - \log p_i^{\star}\right) \left(\sum_j \tilde{p}_j(\alpha) \log p_j^{\star} + \HHs(\bm{p}^{\star})\right) \nonumber\\
    && \left. - \tilde{p}_i(\alpha) \sum_j \tilde{p}_j(\alpha) \log p_j^{\star} \left(\sum_k \tilde{p}_k(\alpha) \log p_k^{\star} - \log p_j^{\star}\right)\right\}\nonumber\\
    &=& \frac{-p_i^{\star} \log p_i^{\star}(\HHs(\bm{p}^{\star}) + \log p_i^{\star})
    +p_i^{\star} \sum_j p_j^{\star} \log p_j^{\star}(\HHs(\bm{p}^{\star}) + \log p_j^{\star})}{2}\nonumber\\
    &=& \frac{-\HHs(\bm{p}^{\star}) p_i^{\star}\log p_i^{\star} - p_i^{\star} (\HHs(\bm{p}^{\star}))^2 - p_i^{\star}\log^2 p_i^{\star} + p_i^{\star} \sum_j p_j^{\star}\log^2 p_j^{\star}}{2}.
\end{eqnarray}
Finally, summing Eq.~\ref{eq:A_shannon} and Eq.~\ref{eq:B_shannon_02}, we get
\begin{eqnarray}\label{eq:gradient_alpha_shannon_02}
    \frac{\partial p_i^{\star}}{\partial \alpha}\bigg|_{\alpha=1} &=& \frac{- p_i^{\star}\log^2 p_i^{\star} + p_i^{\star} \sum_j p_j^{\star}\log^2 p_j^{\star}}{2}.
\end{eqnarray}

\section*{Summary}

To sum up, we have the following expression for the Jacobian of $\aentmax$ with respect to $\alpha$:

\begin{equation}\label{eq:final_gradient_alpha_supp}
    \frac{\partial p_i^{\star}}{\partial \alpha} =
    \left\{
    \begin{array}{ll}
        \frac{p_i^{\star} - \tilde{p}_i(\alpha)}{(\alpha-1)^2} - \frac{p_i^{\star}\log p_i^{\star} + \tilde{p}_i(\alpha)\HHs(\bm{p}^{\star})}{\alpha-1}, & \text{for $\alpha > 1$}  \\
        \frac{- p_i^{\star}\log^2 p_i^{\star} + p_i^{\star} \sum_j p_j^{\star}\log^2 p_j^{\star}}{2},                                                    & \text{for $\alpha = 1$.}
    \end{array}
    \right.
\end{equation}

\cleardoublepage
\singlespacing