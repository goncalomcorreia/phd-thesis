\fancychapter[Conclusions]{Conclusions}
\label{chap:conclusions}

\cleardoublepage
\doublespacing

To wrap up the present thesis, we review the main contributions
developed in this work, and discuss their impact, limitations, and
potential directions for future work.

\section{Summary of Contributions}

Over the course of this thesis, we have developed models and
techniques that aim to make neural models more data-efficient, transparent,
and compact.

In \chapref{chap:ape}, we achieved \textbf{data-efficiency} by
leveraging the \textbf{transfer learning} power of a pre-trained
language model to avoid the need for producing a large synthetic
dataset, in the context of a real-world application: Automatic
Post-Editing.

In \chapref{chap:adaptsparse}, we proposed a model with increased
\textbf{transparency} by letting it \textbf{learn its own sparsity}.
The chosen application was NMT, and we showed
how the sparsity can help to interpret the various roles of each
attention head in the model.

Finally, in \chapref{chap:sparsemarg}, we once again leveraged
\textbf{learnable sparsity} by introducing a new method to train
latent variable models with discrete or structured nodes. We show the
efficacy of our method in several applications: semi-supervised
learning, emergent communication, and unsupervised learning of
bit-vector representations. This method achieves better performance
than standard methodologies for training discrete latent variable
models and as such aids in the development of more \textbf{compact}
neural models.

A common theme in this thesis is sparsity, particularly
sparse probability distributions. We have followed in the footsteps
of work started by \citet{sparsemax} and proposed novel uses of
these sparse distributions and promising alternatives to them.
In all works described, we have shown how, in different applications,
we can let neural models learn how much sparsity they need; be it
in the context of an attention mechanism, or in the context of
the number of relevant latent assignments during training.

On another front, we tackle some forms of \textbf{weak supervision}.
On one hand, we use transfer learning and minimize the amount of data
required to train an APE model in \chapref{chap:ape}. On the other
hand, we also applied our method of \chapref{chap:sparsemarg} to
obtain improved performance on a semi-supervised learning task.
Additionally, in \secref{sec:subsequent_work_adapt} we saw how our
discovered attention head roles can be used to fix attention patterns
in a Transformer, increasing inductive bias in the
process~\citep{raganato2020FixedEncoderSelfAttentiona}.

\section{Open Problems and Limitations}

In the pursuit of diligent research, we briefly discuss some of the
open problems and limitations of the work presented in this thesis.
We seek not to diminish any of these limitations, but rather present
them fairly.
%
% We will also discuss the ethical ramifications of our
% work in a later section (\secref{sec:impact}).

As previously stated, sparsity is a prominent theme in this work.
That sparsity is achieved through probability normalization functions
such as sparsemax, $\alpha$-\entmaxtext, and SparseMAP. While
sparsity can lead to increased efficiency, we do not fully take
advantage of this in the present thesis or in the publicly released
code. For example, for $\alpha$-\entmaxtext, further speed-ups are
possible, with careful engineering, by leveraging more parallelism in
the bisection algorithm for computing it.
This is a relevant research direction, as it is now possible to
exploit recent generations of GPU architectures which are more
sparse-friendly than before.

In \chapref{chap:ape}, we have used a large pre-trained model to
circumvent the need to train MT models to produce a large synthetic
dataset and then train an APE model on it. Although we believe
this to be a way to spare resources, we do not wish to underplay
the resources used to train the pre-trained model itself.

While we have shown promising results in \chapref{chap:sparsemarg},
we are aware that the experiments we conducted were performed on toy
datasets. While this may overemphasize the success of the work, we
have seen in \secref{sec:subsequent} that our method has already been
successfully applied to applications that have real-world datasets.
We hope to continue seeing this trend in the future.

\section{Future Directions}

We further list in this section exciting directions for future work
that draw inspiration from the methods proposed in this thesis. We
hope this might inspire other researchers in the field to explore
these directions and extend the ideas presented.

\paragraph{Semi-supervised Learning.} Some forms of weak
supervision were explored in this work. In particular, we have shown
promising results in \chapref{chap:sparsemarg} by introducing a new
method to train discrete latent variable models, which can be used to
train semi-supervised models (\secref{sec:gen}). We believe that our
method can be used in real-world applications where supervision is
limited and that it will lead to better results than standard
methods. For example, in NLP, there are many applications where a
relaxation of the discrete space is not possible~\citep{Lee2019} and
where we could only rely on high variance methods if we wished to
train a latent variable model. We can now circumvent this limitation
by using our training method.
% TODO: add more details how this can be done

\paragraph{Non-differentiability of the LVM learning signal.} We note
how, in \eqnref{eq:fit}, the training of the parameters that learn
$\pi(z|x;\theta)$ is decoupled from the training of the parameters of
the generative model of the downstream loss: when marginalizing the
latent variables, the downstream loss only scales the gradients
computed from $\pi(z|x;\theta)$. This means that our method can still
be applied when the downstream loss component does not have any
trainable parameters. Furthermore, that component in \eqnref{eq:fit}
can be replaced by any function, even non-differentiable, akin to a
reward signal. One example of a real-world application is prompting
of large pre-trained language models. In this line of work, the
pre-trained model is assumed to already have the knowledge required
to handle the task at hand stored in its parameters and only a prompt
is required to generate the desired output. This problem could be
framed as a latent variable model and interpret the natural language
prompt as a discrete latent variable. Using the property we just
described, the downstream task can be optimized via any
differentiable or non-differentiable reward signal.

% TODO: add idea of latent draft

% TODO: add broader impact of this work

% \section{Broader Impact}\label{sec:impact}

% We discuss here the broader impact of our work.


% Currently, the solutions available to train discrete latent variable
% models (LVMs) greatly rely on MC sampling, which can have high variance.
% Methods that aim to decrease this variance are often not trivial to
% train and to implement and may disincentivize practitioners from
% using this class of models. However, we believe that discrete LVMs
% have, in many cases, more interpretable and intuitive
% latent representations. Our methods offer: a simple approach in
% implementation to train these models; no addition in the number of
% parameters; low increase in computational overhead (especially when
% compared to more sophisticated methods of variance
% reduction~\citep{RB19}); and improved performance. Our code has been
% open-sourced as to ensure it's scrutinizable by
% anyone and to boost any related future work that other researchers
% might want to pursue.

% As we have already pointed out, oftentimes LVMs
% have superior explanatory power and so can aid in understanding cases
% in which the model failed the downstream task. Interpretability of
% deep neural models can be essential to better discover any ethically
% harmful biases that exist in the data or in the model itself.

% On the other hand, the generative models discussed in this work may
% also pave the way for malicious use cases, such as is the case with
% \emph{Deepfakes}, fake human avatars used by malevolent Twitter
% users, and automatically generated fraudulent news. Generative models
% are remarkable at sampling new instances of fake data and, with the
% power of latent variables, the interpretability discussed before can
% be used maliciously to further push harmful biases instead of
% removing them. Furthermore, our work is promising in improving the
% performance of LVMs with several discrete
% variables, that can be trained as attributes to control the sample
% generation. Attributes that can be activated or deactivated at will
% to generate fake data can both help beneficial and malignant users to
% finely control the generated sample. Our work may be currently
% agnostic to this, but we recognize the dangers and dedicate effort to
% combating any malicious applications.

% Energy-wise, LVMs often require less data and
% computation than other models that rely on a massive amount of data
% and infrastructure. This makes LVMs ideal for
% situations where data is scarce, or where there are few computational
% resources to train large models. We believe that better latent
% variable modeling is a step forward in the direction of alleviating
% environmental concerns of deep learning
% research~\citep{strubell2019energy}. However, the models proposed in
% this work tend to use more resources earlier on in training than
% standard methods, and even though in the applications shown they
% consume much less as training progresses, it's not clear if that
% trend is still observed in all potential applications.

% In data science, LVMs, such as
% mixed-membership models~\citep{blei2014build}, can be used to uncover
% correlations in large amounts of data, for example, by clustering
% observations. Training these models requires various degrees of
% approximations which are not without consequence, they may impact the
% quality of our conclusions and their fealty to the data. For example,
% variational inference tends to under-estimate uncertainty and give
% very little support to certain less-represented groups of variables.
% Where such a model informs decision-makers on matters that affect
% lives, these decisions may be based on an incomplete view of the
% correlations in the data and/or these correlations may be exaggerated
% in harmful ways. On the one hand, our work contributes to more stable
% training of LVMs, and thus it is a step towards addressing some of
% the many approximations that can blur the picture. On the other hand,
% sparsity may exhibit a larger risk of disregarding certain
% correlations or groups of observations, and thus contribute to
% misinforming the data scientist. At this point it is unclear to which
% extent the latter happens and, if it does, whether it is consistent
% across LVMs and their various uses. We aim to study this issue
% further and work with practitioners to identify failure cases.

\cleardoublepage

\singlespacing