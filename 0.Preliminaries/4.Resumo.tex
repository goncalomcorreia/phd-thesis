\begin{resumo}

    \noindent Em aprendizagem automática, os modelos baseados em
    redes neuronais tornaram-se omnipresentes no estado da arte. A
    composição destes modelos baseia-se em blocos diferenciáveis que
    dão origem a representações vetoriais densas dos dados
    subjacentes. Para obter boas representações, os métodos
    convencionais requerem o manuseamento de muitos dados. Para além
    disso, embora obtenham excelente desempenho, estes modelos não
    são interpretáveis e não fornecem representações dos dados de
    forma compacta. Para resolver estes problemas, esta tese propõe
    soluções que envolvem \textbf{esparsidade} e várias formas de
    \textbf{supervisão fraca}. Para obter \textbf{eficiência} de
    dados, usamos técnicas de transferência de informação como uma
    forma de supervisão fraca. O modelo proposto tem um desempenho
    semelhante a modelos treinados em milhões de dados, embora tenha
    sido treinado em apenas poucos milhares de exemplos. Para obter
    \textbf{transparência}, propomos uma função de normalização que
    tem a capacidade de aprender a sua própria esparsidade, ou seja,
    capaz de aprender a atribuir valores nulos. Esta função é
    diferenciável e a esparsidade pode ser por isso adaptada de
    acordo com os dados e de acordo com o papel que a componente
    neuronal do modelo em que se insere. Sem custos no desempenho, a
    esparsidade ajuda a descobrir especializações das componentes
    neuronais, ajudando a interpretabilidade de um modelo de tradução
    automática. Para obter \textbf{compacidade}, propomos uma maneira
    de obter gradientes determinísticos de forma eficiente, no treino
    de modelos com variáveis latentes discretas ou estruturadas.
    Estas componentes discretas têm a capacidade de desvendar grupos
    e estruturas inerentes aos dados, compactando por isso a
    informação. No entanto, treinar estes modelos pode ser complexo,
    pois exige aproximações através de amostragem ou relaxamentos
    para o espaço contínuo. Com a técnica utilizada neste estudo,
    obtemos gradientes determinísticos ao parametrizar as
    distribuições com funções esparsas, tanto estruturadas como
    não-estruturadas. Obtemos bom desempenho em três aplicações
    diferentes, alcançando, de qualquer forma, as vantagens práticas
    das aproximações acima mencionadas. Graças a estas novas
    contribuições científicas, a presente tese desafia a doutrina
    atual de que modelos neuronais não são capazes de exibir
    eficiência de dados, transparência, nem compacidade.

    % A presente tese cujo tema central consiste em melhorar a
    % eficiência na utilização de dados, a transparência, e a
    % compacidade de redes neuronais, propõe novos avanços em técnicas de
    % esparsidade e supervisão fraca, defiando a doutrina de que
    % modelos neuronais não são capazes de exibir tais características.

    % Para atingir eficiência na utilização de dados, propomos
    % aproveitar as capacidades de transferência de aprendizagem de
    % redes neuronais pré-treinadas em tarefas sequenciais. Aplicamos
    % este método em Pós-Edição Automática (PEA) para demonstrar a sua
    % eficácia. Para isso, usamos um modelo neuronal pré-treinado nos
    % dois componentes principais do sistema de PEA, explorando várias
    % estratégias de compartilhamento de parâmetros. Conseguimos obter
    % resultados competitivos, quando comparado com modelos treinados
    % em grandes quantidades de dados artificiais, mas utilizando
    % apenas um pequeno conjunto de dados.

    % Para atingir transparência, propomos substituir a função softmax
    % nos mecanismos de atenção de redes neuronais pela função
    % $\alpha$-entmax: uma generalização diferenciável da softmax que
    % permite que palavras com baixa pontuação recebam precisamente um
    % peso de zero. Para além disso, derivamos um método para aprender
    % o parâmetro $\alpha$ automaticamente, que controla a forma e a
    % dispersão da função $\alpha$-entmax. Em tradução automática,
    % mostramos que a nossa abordagem adaptativa melhora a
    % interpretabilidade quando comparado à simples aplicação da
    % softmax.

    % Para atingir modelos compactos, propomos uma nova técnica para
    % treinar modelos com variáveis latentes discretas. Treinar este
    % tipo de modelos é muitas vezes desafiador devido à necessidade de
    % usar estratégias de amostragem, que têm uma variância alta, ou
    % por ter de usar relaxamentos para o espaço contínuo. Em
    % alternativa, propomos usar mapeamentos esparsos diferenciáveis
    % para parametrizar as distribuições discretas. A esparsidade
    % permite que se faça uma marginalização eficiente das variáveis
    % latentes, em vez da utilização das técnicas mencionadas. Usamos o
    % nosso método em três diferentes aplicações e obtemos resultados
    % competitivos atingindo, mesmo assim, a praticabilidade de
    % aproximações baseadas em amostragem.

\end{resumo}
