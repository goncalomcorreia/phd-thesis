\newgeometry{top=2.5cm, bottom=2.5cm, inner=2.5cm, outer=2.5cm}
\onehalfspacing
\begin{abstract}

    This thesis proposes new techniques for improving neural network
    models' data efficiency, transparency, and compactness, primarily
    applied to natural language processing. Thanks to our proposed
    advances in sparsity and weak supervision, we challenge the
    belief that neural models are not capable of exhibiting such
    characteristics.

    For data efficiency, we propose to leverage the transfer learning
    capabilities of large neural encoder models into data-hungry
    sequence-to-sequence tasks. We apply our method to Automatic
    Post-Editing (APE) to demonstrate its effectiveness. We fine-tune
    a pre-trained neural encoder model on both the encoder and
    decoder of an APE system, exploring several parameter sharing
    strategies. We obtain competitive results with models trained on
    larger artificial datasets by training on only a small dataset.

    For transparency, we propose to replace the softmax on attention
    mechanisms of neural models with $\alpha$-entmax: a
    differentiable generalization of softmax that allows low-scoring
    words to receive precisely zero weight. Moreover, we derive a
    method to learn the $\alpha$ parameter automatically---which
    controls the shape and sparsity of $\alpha$-entmax. On neural machine
    translation, we show that our adaptive approach improves
    interpretability when compared to \textit{vanilla} softmax
    approaches.

    For compactness, we propose a new technique to train discrete
    latent variable models more effectively. Traning these models is
    often challenging due to needing to rely on high-variance
    sampling strategies or continuous relaxations. Instead, we propose to use
    differentiable sparse mappings to parameterize the latent
    discrete distributions. Sparsity will enable efficient
    marginalization as an alternative to previous techniques. We
    obtain good performance on three latent variable model
    applications while still achieving the practicality of
    sampling-based approximations.

\end{abstract}