\begin{itemize}
    \item backpropagation; define backpropagation
    \item sequence modelling; define the task of sequence modeling
    \item sequence-to-sequence; define scaled attention
    \item transformers; define multi-head attention
    \item pre-trained language models: ELMo, BERT (encoders), GPT (language model), T5 (seq2seq)
\end{itemize}

\section{Machine Learning}
\label{sec:ml-primer}

We begin by presenting the main ingredients in the machine learning pipeline:
%
\begin{itemize}
    \item a dataset $\mathcal{D}$, made out of
          \textit{inputs} $\mathcal{D}=\left\{\bm{x}_n\right\}_{n=1}^N$ or
          \textit{input-output} pairs
          $\mathcal{D}=\left\{(\bm{x}_n,y_n)\right\}_{n=1}^N$;
    \item a set of parameters $\theta$;
    \item a loss function $L$.
\end{itemize}
%
With these ingredients, we create a machine learning pipeline by
taking a \textit{training set} that is a subset of $\mathcal{D}$ and estimate
the parameters $\theta$ by minimizing the loss
$L$ on that set. The resulting model with optimal
$\hat{\theta}$ can then be evaluated on a held-out set of
$\mathcal{D}$, called \textit{test set}, to assess its performance.

From a probabilistic perspective, this process creates a probability model
that identifies the probability measure of a random experiment:
it maps from available inputs in the set of possible inputs $\mathcal{X}$ to a probability
distribution of a random variable $Y$ (when $\mathcal D$ is made of input-output pairs).
When the set of possible outputs $\mathcal{Y}$ is discrete, the probability
distribution of $Y$ is prescribed by a probability mass function (pmf) $p_{Y|X}(y|x;\theta)$;
when the set of possible outputs $\mathcal{Y}$ is continuous, the probability
distribution of $Y$ is prescribed by a probability density function (pdf).
In the context of the present thesis, we will not have applications in which
$\mathcal{Y}\in\reals$ (\ie \textit{regression problems}) and as such,
when referring to $p_{Y|X}(y|x;\theta)$, we assume that it is a
probability mass function. Furthermore, our problems will
most often involve a $k$-way classification, and so
$p_{Y|X}(y|x;\theta)$ is specifying a \textit{Categorical} distribution,
%
\begin{equation}
    Y|X\!\!=\!x \sim \text{Cat}(f(x; \theta)),
\end{equation}
%
where $f(x; \theta)$ is a function such that
$p_{Y|X}(y|x;\theta)=\text{Cat}(y|f(x; \theta))$.

\paragraph*{Supervised learning.} We now turn our attention to applications
in which one uses a dataset of input-output pairs. In such a case,
to estimate $\theta$,
%
\begin{equation}
    \hat{\theta} = \argmin_\theta L_\mathcal{D}(\theta),
\end{equation}
%
where $L_\mathcal{D}(\theta)$ is the loss function:
the \textit{negative log-likelihood}
%
\begin{equation}
    L_\mathcal{D}(\theta) = -\mathcal{L}_\mathcal{D}(\theta) =
    - \sum_n \log p_{Y|X}(y_n|x_n; \theta).
\end{equation}
%
Typically, the loss function might also entail a regularization term,
which is added to the negative log-likelihood
%
\begin{equation}
    L_\mathcal{D}(\theta) = -\mathcal{L}_\mathcal{D}(\theta) + \mathcal{R}(\theta).
\end{equation}
%
Sometimes, instead of the probabilistically-grounded negative
log-likelihood, practitioners might use a non-probabilistic loss and
still use a statistical model; for example, one might use the 0/1
loss where the loss is $1$ if the prediction $\hat{y}\neq y$ and $0$
otherwise.
After training the model, we
wish to make predictions $\hat{y}$ for each novel input $\bm{x}$, and we can
make such predictions by using the optimal trained parameters $\hat{\theta}$
and selecting the highest-scoring $y\in\mathcal{Y}$
%
\begin{equation}
    \hat{y}=\argmax_{y\in\mathcal{Y}}p_{Y|X}\left(y|\bm{x};\hat{\theta}\right).
\end{equation}
%
For example, in a binary sentiment classification problem of English
sentences, $\mathcal{Y}=\{\text{negative},\text{positive}\}$, and
$\mathcal{X}$ would be the set of possible sentences in the English
language.

% TODO: not a distribution! pmf
% TODO: "likelihood of the model" NOT "likelihood of the data."

\paragraph*{Unsupervised learning.} Similarly, we can also use just a
dataset of inputs $\bm{x}$ and estimate the parameters $\theta$ under a loss to
obtain an unsupervised model; that is, a model in which the predictions are
not based on any labels. In this case, the loss function $L$ only
depends on the inputs $\bm{x}_n$ and the parameters $\theta$
and we thus model $p_X(\bm{x}|\theta)$.
In this case, the loss function is
%
\begin{equation}
    L_\mathcal{D}\left(\theta\right)=-\sum_n\log p_X\left(\bm{x}_n|\theta\right).
\end{equation}
%
In this setting, to assess the resulting model's performance,
we often compute the log-likelihood of the model under the test set
%
\begin{equation}
    \mathcal{L}_{\mathcal{D}_\text{test}}(\hat{\theta}) =
    \sum_n\log p_X\left(\bm{x}_n|\hat{\theta}\right).
\end{equation}
%
that is, we assess the likelihood of the model being able to generate
the data points $\bm{x}$.

\paragraph*{Semi-supervised and weakly-supervised learning.} In a
semi-supervised setting, there is tipically a smaller portion of the
dataset that is labeled,
$\mathcal{D}_{\text{labeled}}=
    \left\{(\bm{x}_n,y_n)\right\}_{i=1}^{N_{\text{labeled}}}$,
and the remaining majority is unlabeled,
$\mathcal{D}_{\text{unlabeled}}=
    \left\{(\bm{x}_n)\right\}_{i=1}^{N_{\text{unlabeled}}}$.
The parameters are then estimated by using a mixture of losses: a
component that is supervised and a component that is unsupervised.
Taking the loss examples given above, a semi-supervised model could
have the loss function
%
\begin{equation}
    L_{\mathcal D}\left(\theta\right)=
    -\sum_n\log p_{X}\left(\bm{x}_n|\theta\right)
    -\sum_n\mathbbm{1}\left[(\bm{x}_n,y_n)\in\mathcal{D}_{\text{labeled}}\right]
    \log p_{Y|X}\left(y_n|\bm{x}_n;\theta\right),
    \label{eq:semisup}
\end{equation}
%
where
$\mathbbm{1}\left[(\bm{x}_n,y_n)\in\mathcal{D}_{\text{labeled}}\right]$
denotes the indicator function that is $1$ if the corresponding
input-output pair exists in the labeled set. In some applications,
$p_X$ and $p_{Y|X}$ are independent components that share some
parameters; in others, they are marginals of the same joint
distribution specified by $p(x, y;\theta)$. In the context of this
thesis, we will use the term \textit{weakly-supervised} and
\textit{weak supervision} to refer to a broader setting that aims to
alleviate the need for labeled data to obtain a well-performing
model. This setting not only includes semi-supervision but also
includes, for example, the transferring of learned parameters of an
already optimized unsupervised model to a supervised one, also called
\textit{transfer learning}. In such a case, the estimation of
parameters can first occur by minimizing the first loss component in
\eqnref{eq:semisup}, and then those learned parameters can be reused
as an initialization on the minimization of the second component.
While not explored in this thesis, other forms of weak supervision
include the usage of underspecified or unreliable labels, linguistic
constraints, labels obtained via heuristics, among others.

\subsection{Linear and Deep Models}

Naturally, there are many different ways one can use $\theta$ to
parameterize $p\left(y|\bm{x};\theta\right)$. One of the simpler ways
is to define a linear model
%
\begin{equation}
    p\left(y|\bm{x};\theta\right)=W_{\theta}^T\bm{\phi}\left(\bm{x}\right),
    \label{eq:linear}
\end{equation}
%
% softmax is how to parameterize the probability mass function as a Categorical
% add softmax
where $\bm{\phi}$ is a function that maps the inputs $\bm{x}$ into a
manageable space, and $W_{\theta}$ is the matrix of weights. The
problem of obtaining the optimal $W_{\theta}$ lies within the set of
problems that convex optimization can solve, which significantly
simplifies the optimization process. While the optimization
simplicity of linear models is appealing, constructing $\bm{\phi}$
may require a significant effort. The creation of a suitable function
$\bm{\phi}: \mathcal{X} \rightarrow \reals^d$ might demand extensive
knowledge of the domain of $\mathcal{X}$ and of the task itself. The
\textit{art} of creating these functions $\bm{\phi}$ is often called
\textbf{feature engineering}.

An alternative to the linear model is the neural network, which, in a
way, does feature engineering automatically through the use of hidden
representations. Neural networks use non-linear functions, also known
as \textit{activation functions}, along with linear functions similar
to \eqnref{eq:linear} in order to build these hidden representations.

\begin{definition}[Neural Network]
    A neural network with parameters $\theta$ is a function
    composed of building blocks that comprise both linear and non-linear
    functions. These building blocks are called \textit{layers}.
    The layers connect in a sequence, and the output of each
    layer is fed into the next layer. The output of the last layer is
    the output of the neural network. Therefore,
    %
    \begin{equation}
        p\left(y|\bm{x};\theta\right)=l_\theta^{n}\left(l_\theta^{n-1}\left(\dots\left(l_\theta^{1}\left(\bm{x}\right)\right)\right)\right)
    \end{equation}
    %
    defines a neural network with $n$ layers, where $l_n$ is the
    $i$-th layer.
\end{definition}

Thanks to its non-linearity, neural networks effectively create
intermediate abstractions of the data optimized to
represent the data more meaningfully in order to succeed at the task, bypassing the need
for feature engineering. Just as well, non-linearities end up
impeding the usage of convex optimization, and so neural networks
are often trained instead through a process called \textit{backpropagation},
a gradient-based optimization algorithm.

\begin{definition}[Backpropagation]
    Backpropagation is
\end{definition}


\section{Neural Networks and Natural Language}

\subsection{Sequence-to-Sequence Models}
\label{sec:transformer_bg}

% Y_i | Y_<i, X

Sequence-to-sequence (seq2seq) models are the broad, main family of models
used to train NMT systems. While there are a plethora of different architectures,
seq2seq models have in common the following blocks:

\begin{itemize}
    \item the {\bf encoder}, which takes as input a sequence $x=[x^1,
              \dots, x^N]$ and creates hidden representations $h=[h^1, \dots,
              h^N]$, by learning representations of each word in $x$ and then
          representations that take into account the context around each word;
    \item the {\bf decoder}, which takes the hidden representations
          $h$ of the encoder and learns a mapping to produce a
          variable-length sequence $\hat{y}$ conditioned on $h$, by
          learning from the true target sentence $y=[y^1, \dots, y^M]$.
\end{itemize}

In the following section, we will expand on the current
state-of-the-art seq2seq architecture.

\subsection{Transformer}

The Transformer~\citep{vaswani2017attention} is a seq2seq model which
maps an input sequence to an output sequence through many layers of
\textbf{multi-head attention} mechanisms, yielding a dynamic,
context-dependent strategy for propagating information within and
across sentences. It contrasts with previous seq2seq models, which
usually rely either on costly gated recurrent operations~\citep[often
    LSTMs:][]{bahdanau2014neural,luong2015effective} or static
convolutions~\citep{convseq}.

Given $n$ query contexts and $m$ sequence items under consideration,
attention mechanisms compute, for each query, a weighted
representation of the items. The particular attention mechanism used
in \citet{vaswani2017attention} is called \emph{scaled dot-product
    attention}, and it is computed in the following way:
%
\begin{equation}
    \att(\bm{Q}, \bm{K}, \bm{V}) = \amap
    \left(\frac{\bm{Q}\bm{K}^\tr}{\sqrt{d}}\right) \bm{V},
    \label{eq:att_scaled_dot}
\end{equation}
%
where $\bm{Q} \in \reals^{n \times d}$ contains
representations of the queries, $\bm{K}, \bm{V} \in \reals^{m \times
        d}$ are the \emph{keys} and \emph{values} of the items attended over,
and $d$ is the dimensionality of these representations. The $\amap$
mapping normalizes row-wise using \textbf{softmax},
$\amap(\bm{Z})_{ij} = \softmax(\bm{z}_i)_j$.

\begin{definition}
    The normalization function \textbf{softmax} is defined as
    \begin{equation}\label{eq:softmax}
        \softmax(\bm{z}) = \frac{\exp(z_j)}{\sum_{j'} \exp(z_{j'})}.
    \end{equation}
\end{definition}

In words, the \emph{keys} compute a relevance score between each item
and query. Then, softmax normalizes these attention weights, and
these will weight the \emph{values} of each item at each query
context.

However, for complex tasks, different parts of a sequence may be
relevant in different ways, motivating \emph{multi-head attention} in
transformers. Multi-head attention is simply the application of
\eqnref{eq:att_scaled_dot} in parallel $H$ times, each with a
different learned linear transformation that allows specialization:
%
\begin{equation}\label{eq:head}%
    \hspace{-.01ex}%
    \ath_i(\bm{Q}\!,\!\bm{K}\!,\!\bm{V})\!=\!\att(\bm{QW}_i^Q\!\!,\bm{KW}_i^K\!\!,\bm{VW}_i^V\!)%
    \hspace{-1.5ex}%
\end{equation}

In the Transformer, there are three separate multi-head attention mechanisms for
distinct purposes:
\begin{itemize}
    \item \textbf{Encoder self-attention:} builds rich, layered representations of
          each input word by attending to the entire input sentence.
    \item \textbf{Context attention:} selects
          a representative weighted average of the encodings of the input words at each
          time step of the decoder.
    \item \textbf{Decoder self-attention:} attends over the partial output sentence
          fragment produced so far.
\end{itemize}
Together, these mechanisms enable the contextualized flow of information between
the input sentence and the sequential decoder.

% \subsection{Large Pre-trained Language Models}
