% Defines an additional alphabet... not required in most cases
% ------------------------------------------------------------
% \DeclareMathAlphabet{\mathpzc}{OT1}{pzc}{m}{it}

% PACKAGE babel:
% ---------------
% The 'babel' package may correct some hyphenisation issues of latex. 
% However in most situations it is not required.
\usepackage[english,portuguese]{babel}


% PACKAGE fontenc:
% -----------------
% chooses T1-fonts and allows correct automatic hyphenation.
%\usepackage[T1]{fontenc}
%\usepackage[latin1]{inputenc}
%\usepackage[utf8]{inputenc}
%\usepackage{lmodern} %will change font type

% Package ulem.
\usepackage{ulem} % Allows the use of other text emphatizer commands
\normalem %defines \emph{} to italic, instead of underline. 
\raggedbottom %declaration makes all pages the height of the text on that page. No extra vertical space is added. The \flushbottom declaration makes all text pages the same height, adding extra vertical space when necessary to fill out the page.

% PACKAGE date time:
% -----------------
% Lets you alter the format of the date that \today returns.
\usepackage{datetime}
\newdateformat{todaythesis}{%
  \monthname[\THEMONTH]  \THEYEAR}

% PACKAGE latexsym:
% -----------------
% Defines additional latex symbols. May be required for thesis with many math symbols.
\usepackage{latexsym}

\newcommand\hmmax{0}
\newcommand\bmmax{0}
% MATH PACKAGES amsthm, amssymb, amsfonts...:
% -------------------------------------------
% This package is typically required. Among many other things it adds the possibility
% to put symbols in bold by using \boldsymbol (not \mathbf); defines additional 
% fonts and symbols; adds the \eqref command for citing equations. I prefer the style
% "(x.xx)" for referering to an equation than to use "equation x.xx".
\usepackage{amsthm}
\usepackage{amssymb}
\usepackage{amsfonts}
\usepackage{amsbsy}
\usepackage{mathtools}%The mathtools package fixes some amsmath quirks and adds some useful settings, symbols, and environments to amsmath.
\usepackage[dvipsnames]{xcolor}
\usepackage{cancel} % xcolor to change the texto color; cancel to use a cancel line in equations.
%https://en.wikibooks.org/wiki/LaTeX/Colors

% PACKAGE TABLES multirow, colortbl, longtable:
% ---------------------------------------
% These packages are most usefull for advanced tables. The first allows to join rows 
% throuhg the command \multirow which works similarly with the command \multicolumn
% The second package allows to color the table (both foreground and background)
% The third package is only required when tables extend beyond the length of one page;
% with compatibilities with the tabular environment. The last allow the definitions of landscape pages, allowing the use of a different orientation for wider graphics or tables. See package documentation to see the implementation.
\usepackage{multirow}
\usepackage{colortbl}
\usepackage{supertabular}
\usepackage{pdflscape}
% \usepackage{longtable}
\usepackage{tabularx}	% (default: necessary for the cover)
\usepackage{longtable,tabu}

% PACKAGE GRAPHICS graphics, epsfig, caption, etc...:
% ---------------------------------------------
% Packages for figures... well you will certainly need these packages, with the exception
% of the 'caption' package. This only allows to define extra caption options.
% Notice that subfigure allows to place figures within figures with its own caption. It
% should be avoided to create an eps file with subfigures. That will mean that you won't be 
% able to reference those subfigures. Instead create an EPS file (the only graphics format supported
% by latex) for each of the subfigures and then use the command \subfigure (see below).
\usepackage{graphics}
\usepackage{graphicx}
%\usepackage[pdftex]{graphicx} %> esta selecção provoca colisão com epsfig e caption...
\usepackage{epsfig}	%colisao com graphicx
%to alter captions  (\usepackage[footnotesize,bf,center]{caption})
\usepackage[font=normal,labelfont=bf,textfont=normalfont]{caption}
%\usepackage[hang,small,bf]{subfigure} %deprecated > use subfig or subcaption:
\usepackage{subcaption}	%for subfigures
\usepackage{dcolumn}
\usepackage{bm}
\usepackage{booktabs}
\usepackage{rotating}

\usepackage{color}% to alter text colour
\usepackage{pstricks} % to inkscape latex... not working
\usepackage{import} %to pdf-tex import from different place

% PACKAGE algorithmic, algorithm
% ------------------------------
% These packages are required if you need to describe an algorithm.
% \usepackage{algorithmic}
% \usepackage[chapter]{algorithm}

% PACKAGE natbib/cite/biblatex
% -------------------
% The three packages are not compatible, and you should use one of the two. Notice however that the
% IEEE BiBTeX stylesheet is imcompatible with the natbib package. If using the IEEE format, use the 
% cite package instead
%% Natbib
%\usepackage[square,numbers,sort&compress]{natbib}
\usepackage{natbib}
%% Cite
%\usepackage{cite}
%% Biblatex (Not working)
%\usepackage{csquotes}
%\usepackage[backend=biber,style=authoryear]{biblatex}


% PACKAGE acronyum
% -----------------
% This package is most useful for acronyms. The package guarantees that all acronyms definitions are 
% given at the first usage. IMPORTANT: do not use acronyms in titles/captions; otherwise the definition 
% will appear on the table of contents.
\usepackage[printonlyused]{acronym}
\usepackage[titletoc,title,header]{appendix}
\usepackage[noauto]{chappg}
%The following line assures that each chapter deals with acronyms idependently.
%You may also do it manually by calling acresetall anywhere in the documento to reset the acronyms behaviour. 
\preto\chapter\acresetall

% PACKAGE extra_functions VER COMO DEVE SER
% -----------------
% My Personal package: defines the following commands:
% \fancychapter{chaptername) -> Prints a fancier chapter (you can also use the fancychapter package for this)
% \hline{width} -> use for a replacement of the \hline command
% \Mark1, \Mark2, \Mark3, ...
\usepackage{00.extra_functions}
\usepackage{xr-hyper}
\usepackage{float}
%\usepackage[final]{00.listofsymbols}
\usepackage{00.symlist}

% Set paragraph counter to alphanumeric mode
% \renewcommand{\theparagraph}{\Alph{paragraph}~--}
\renewcommand{\theparagraph}{}

\usepackage{enumitem}	% para set list
\usepackage{algpseudocode} % algorithmic pseudocode
\usepackage[chapter]{algorithm}

\usepackage{scrextend}
% \usepackage{marginnote}%temporary notes at the margin
\usepackage{multicol} %multiple columns environment
%solves figure problem: http://tex.stackexchange.com/questions/12262/multicol-and-figures
\newenvironment{Figure}
{\par\medskip\noindent\minipage{\linewidth}}
{\endminipage\par\medskip}

\usepackage{siunitx}
% \sisetup{load-configurations = abbreviations}
%%% -------------------- declares extra units
\DeclareSIUnit\Mtoe{\mega \tonne oe} % use: \SI{100}{\Mtoe}
\DeclareSIUnit\GWh{GWh} % use: \SI{100}{\GWh}  (not \GW\hour = GW h )
%%% -------------------------------------------------

\usepackage{wrapfig} % to wrap text around small figures
\usepackage{dcolumn} %dcolumns in tables : https://en.wikibooks.org/wiki/LaTeX/Tables#Defining_multiple_columns

\usepackage{makeidx} %necessary?

% PACKAGE hyperref
% -----------------
% Set links for references and citations in document
% Some MiKTeX distributions have faulty PDF creators in which case this package will not work correctly
% Long live Linux :D
\usepackage[hidelinks,plainpages=false,breaklinks]{hyperref}
\hypersetup{
  colorlinks=true,  % false for printing!!!
  linkcolor=linksgreen,
  citecolor=linksgreen,
  urlcolor=linksgreen,
  breaklinks=true,
  bookmarksnumbered=true,
  bookmarksopen=true,
  pdftitle={Learnable Sparsity and Weak Supervision for Data-Efficient, Transparent, and Compact Neural Models},
  pdfauthor={Gonçalo Migueis de Matos Afonso Correia},
  pdfsubject={Doctoral Thesis in Electrical and Computer Engineering},
  pdfcreator={Gonçalo Migueis de Matos Afonso Correia},
  pdfkeywords={thesis,phd,sparsity,compact,transparent,efficient,neural,networks,machine,learning}}
%\usepackage[pdftex,bookmarks,colorlinks,breaklinks]{hyperref}  % PDF hyperlinks, with coloured links
%%\hypersetup{linkcolor=blue}
%\hypersetup{linkcolor=black}

\usepackage{autonum} % only number eqs if they are referenced

%%%%%%%%%%%%

\usepackage{times}
\usepackage{etoolbox}
\usepackage{nicefrac}
\usepackage{empheq}
\usepackage{xurl}
\usepackage{xspace}
\usepackage{enumitem}
\usepackage{tikz}
\usepackage{stmaryrd}
\SetSymbolFont{stmry}{bold}{U}{stmry}{m}{n}

\usepackage{lscape}

\usepackage{tocloft}

\usepackage{setspace}

% fonts
% garamond
% \usepackage{ebgaramond}
% baskerville
% \usepackage{librebaskerville}
\usepackage[T1]{fontenc}
\usepackage[p]{baskervillef}  % [osf] for old style
\usepackage[varqu,varl,var0]{inconsolata}
\usepackage[scale=0.95,type1]{cabin}

\usepackage[baskerville,vvarbb]{newtxmath}
\usepackage[cal=boondoxo]{mathalfa}

\usepackage{anyfontsize}
\usepackage[fontsize=12pt]{fontsize}

% mdframe for definitions
\usepackage{mdframed}
