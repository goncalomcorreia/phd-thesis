% %%%%%%%%%%%%%%%%%%%%%%%%%%%%%%%%%%%%%%%%%%%%%%%%%%%%%%%%%%%%%%%%%%%%%%
% The Introduction:
% %%%%%%%%%%%%%%%%%%%%%%%%%%%%%%%%%%%%%%%%%%%%%%%%%%%%%%%%%%%%%%%%%%%%%%
\fancychapter[Introduction]{Introduction}
\label{cap:int}

\cleardoublepage
\doublespacing

\noindent Deep learning is a field of machine learning that uses
\textbf{neural networks} to learn and obtain predictions
and inferences from unprocessed data. The success of well-known deep
learning models~\citep[\textit{inter
        alia}]{convnet,devlin2018bert,brown2020language} relies on a rich
parameterization accomplished through the use of numerous layers of
computation. Along with vast amounts of data points,
such heavy parameterization allows these models to learn good
vector representations that permit them to excel in their respective
tasks. However, these models are often highly overparameterized;
additionally, their interpretation is difficult due to the underlying
opaqueness of neural models. Moreover, they require costly computing
power, which causes environmental concerns~\citep{Strubell2019}. In
natural language processing (NLP), one such model is the transformer
architecture~\citep{vaswani2017attention} which has quickly risen to
prominence thanks to its outstanding performance, leading to
improvements in the state-of-the-art of neural machine
translation~\citep[NMT;][]{marian,ott2018scaling}, and served as an
inspiration to even bigger and more powerful general-purpose models
like BERT~\citep{devlin2018bert} and
\mbox{GPT-3}~\citep{brown2020language}.

On the other hand, neural latent variable models are powerful and
expressive tools for finding patterns in high-dimensional data, such
as images or text~\citep{Kim2018,Kingma+2014:VAE,RezendeEtAl14VAE}.
These models have powerful structural biases that guide the model's
training; of particular
interest are \emph{discrete} latent variables, which can recover
categorical and structured encodings of hidden aspects of the data,
leading to compact representations~\citep{KingmaEtAl2014SSVAE} and,
in some cases, superior explanatory power~\citep{titov2008joint,
    Bastings2019}. However, more often than not, training models with
discrete latent nodes is a difficult task due to the need to rely on
high-variance methods~\citep{Williams1992} or relaxations into the
continuous space~\citep{GumbelSoftmax,Concrete}.

In this thesis, we will address the issues mentioned above of overuse
of data points, opaqueness, and the difficulties in training compact
versions of neural models. To obtain the solutions presented, we will
rely on forms of \textbf{weak supervision} and, most importantly, on
\textbf{sparse} projections onto probability spaces.

When this project began in 2018, the use of large pre-trained
language models was very much in an infant state.
ELMo~\citep{peters2018deep} had just been released, closely followed
by BERT~\citep{devlin2018bert}, and practitioners were just starting
to use these models in their research to improve the state-of-the-art
of tasks in NLP. The transformer
arquitecture~\citep{vaswani2017attention} was also a promising model
that had just started to pick up much steam and to replace the
recurrent neural network (RNN) sequence-to-sequence models that were
prominent in the NLP literature at the
time~\citep{bahdanau2014neural}. During the course of this project,
there has been remarkable progress in neural networks and NLP
research; for example, pre-trained language model literature has
evolved from proposing simply encoders that provide rich contextual
representations~\citep{devlin2018bert}, to also provide decoders that
deliver extremely realistic text~\citep[GPT-3;][]{brown2020language},
and encoder-decoder models that allow for any task to be posed as a
natural language
prompt~\citep{raffel2020Exploringlimitstransfer,lewis2020BARTDenoisingSequencetoSequence}.

The approaches found in the present thesis are, in part, a product of
the aforementioned research efforts. As we will see in the following
chapters, we introduced a way to use large pre-trained encoders in
generation tasks before large pre-trained decoder and encoder-decoder
models were developed; we tackled the opaqueness of the, at the time,
recently proposed and barely understood transformer architecture; and
pioneered a new paradigm in discrete latent variable model training.

We also note that this thesis has strong roots in a line of
literature that has been proposing new methods to induce sparsity
within the computational graphs of neural networks. Before this
thesis, the foundations had been laid by the study of sparse
normalization functions~\citep{sparsemax,fusedmax,entmax} and their
applications~\citep{maruf2019selective,malaviya2018sparse}, and also
their uses in structured
prediction~\citep{niculae2018sparsemap,sparsemapcg}. During the
course of this project, we have extended this line of work by
tackling some of the abovementioned issues with our sparsity-induced
solutions. We have also created new foundations for others to extend
upon our work and create their own sparse approaches to new and
challenging problems~\citep{treviso2021PredictingAttentionSparsity,
    farinhas2022SparseCommunicationMixed}.

\section{Contributions and Thesis Statement}
\label{sec:int_contributions}

\noindent We will now summarize the contributions of this thesis, which will
address the open questions left to answer in the previous section.

\begin{itemize}

      \item \textbf{We use weak supervision by leveraging transfer learning
                  for data-efficient sequence-to-sequence models.} We show how to
            leverage a pre-trained encoder to perform a
            sequence-to-sequence task on a tiny dataset. We explore different
            avenues of parameter sharing and initialization to make this
            possible.


      \item \textbf{We propose a new deep model with attention that can
                  dynamically adapt to be denser or sparser as needed.} We change the
            transformer architecture such that each attention head, the main
            building block of transformers, can dynamically change its sparsity
            during training. This way, each attention head can accommodate its
            sparsity to its role in the overall model. To achieve that, we derive
            the gradient of $\alpha$ in $\alpha$-\entmaxtext{}~\citep{entmax}, a
            function akin to softmax, in which the sparsity of the probability
            vector is controlled by a parameter $\alpha$.

      \item \textbf{We conduct an extensive analysis on the increased
                  transparency of transformer models that use {\boldmath
                              $\alpha$}-\entmaxtext{} as its normalization function in the
                  attention mechanism.}
            Besides studying the distribution of sparsity and respective $\alpha$ values
            throughout all attention heads in the transformer, we also identify
            examples of sharper attention head behavior than what was found in
            previous work, along with the disentanglement of newfound behaviors
            thanks to our proposed sparsity.

      \item \textbf{We propose a new method to train discrete and
                  structured latent variable models based on the efficiency of
                  marginalizing expectations using sparsity.} Thanks to this method, we
            can train these families of latent variable models without recurring
            to any sampling or relaxations of the discreteness into
            the continuous space. In particular, in the unstructured case, our
            method relies on the sparsemax~\citep{sparsemax} projection function.
            In the structured case, we propose to either use
            SparseMAP~\citep{niculae2018sparsemap} or the novel top-$k$
            sparsemax.

      \item \textbf{We provide open-source code for each of the
                  methods we have proposed.}
            The respective repositories can be found in each
            of the chapters.


\end{itemize}

\paragraph*{Thesis Statement.} The primary claim of this thesis is
that, unlike what numerous previous works insist, neural
models have the capability of being data-efficient, transparent, and
compact: one only needs to look through a different lens that is
capable of leveraging weak supervision, sparsity, and latent
representations. We conclude that a \textit{vanilla}
application of neural models to the problem at hand is not sufficient
for achieving these properties: for \textbf{data-efficiency}, we find that we need
to allow parameter sharing, careful initialization, and powerful
transfer learning capabilities to succeed at a low-resource
generation task; for \textbf{transparency}, we can allow the model to have
sparsity and let it learn it according to its needs at training time; and
for \textbf{compactness}, we can leverage discrete latent variable models in a
better way than what was previously possible by using an exact but
efficient gradient.

\section{Publications}
\label{sec:int_publications}

This thesis is based, in part, on the following publications:

\begin{sloppypar}
      \begin{itemize}

            \item {\bf A Simple and Effective Approach to Automatic
                  Post-Editing with Transfer Learning}~\citep{Correia2019}.
                  Described in \chapref{chap:ape}, this paper was accepted as a
                  poster at ACL 2019.
                  
            \item {\bf Unbabel's Submission to the WMT2019 APE Shared Task:
                  BERT-based Encoder-Decoder for Automatic
                  Post-Editing}~\citep{lopes2019unbabels}. Not included as a specific
                  chapter within this thesis, this work describes an Automatic
                  Post-Editing model that was submitted to the APE Shared Task at
                  WMT2019. This model won the Shared Task, obtaining state-of-the-art
                  in APE at the time.
                  
            \item {\bf Adaptively Sparse
                  Transformers}~\citep{correia2019adaptively}. Described in
                  \chapref{chap:adaptsparse}, this paper was accepted for an oral
                  presentation at EMNLP 2019.
                  
            \item {\bf Efficient Marginalization of Discrete and Structured
                  Latent Variables via Sparsity}~\citep{correia2020procneurips}.
                  Described in \chapref{chap:sparsemarg}, this work was
                  accepted as a spotlight paper at NeurIPS 2020.
                  
      \end{itemize}
\end{sloppypar}
\section{Roadmap}
\label{sec:int_roadmap}

Herein we show the outline of the remainder of this thesis.

We begin in \chapref{chap:background} by reviewing major concepts
that are essential to understand the content of this thesis,
particularly neural network models for NLP, sparse probability
normalization functions, and latent variable models.

In \chapref{chap:ape}, we develop a sequence-to-sequence model for
Automatic Post-Editing using the Transformer architecture and by
harnessing the transfer learning power of a pre-trained large
language model.

In \chapref{chap:adaptsparse}, we augment the Transformer to use
attention modules that allow sparse probabilities that adaptively
change their own sparsity depending on their role within the model.

In \chapref{chap:sparsemarg}, we propose a new training method of
discrete and structured neural latent variable models, which uses
sparse probabilities in order to compute the training objective of
these models, instead of using Monte Carlo methods.

Finally, in \chapref{chap:conclusions}, we summarize the
contributions of the present thesis, address some of the limitations
and open problems of the present work, and discuss exciting future
directions of further research.


\cleardoublepage
\singlespacing
