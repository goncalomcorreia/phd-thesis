% %%%%%%%%%%%%%%%%%%%%%%%%%%%%%%%%%%%%%%%%%%%%%%%%%%%%%%%%%%%%%%%%%%%%%%
% The Introduction:
% %%%%%%%%%%%%%%%%%%%%%%%%%%%%%%%%%%%%%%%%%%%%%%%%%%%%%%%%%%%%%%%%%%%%%%
\fancychapter{Introduction}
\label{cap:int}

\doublespacing

The success of well known deep learning
models~\citep{convnet,devlin2018bert,brown2020language} rely on a
rich parameterization through the use of numerous neural network
layers. Along with huge amounts of datapoints, this allows for these
models to learn good vector representations that consequently permit
these models to excel in their respective tasks. However, these
models are often highly overparameterized and their interpretation is
difficult. Moreover, they require very expensive computing power,
which causes understandable environmental
concerns~\citep{Strubell2019}. In natural language processing (NLP),
one such model is the Transformer
architecture~\citep{vaswani2017attention} which has quickly risen to
prominence through its performance, leading to improvements in the
state of the art of neural machine
translation~\citep[NMT;][]{marian,ott2018scaling}, and served as
inspiration to even bigger and powerful general-purpose models like
BERT~\citep{devlin2018bert} and
\mbox{GPT-3}~\citep{brown2020language}.

On the other hand, neural latent variable models are powerful and
expressive tools for finding patterns in high-dimensional data, such
as images or text~\citep{Kim2018,Kingma+2014:VAE,RezendeEtAl14VAE}.
These models have powerful structural biases that guide the model's
training, leading to models that are more transparent. Of particular
interest are \emph{discrete} latent variables, which can recover
categorical and structured encodings of hidden aspects of the data,
leading to compact representations~\citep{KingmaEtAl2014SSVAE} and,
in some cases, superior explanatory power~\citep{titov2008joint,
      Bastings2019}.

Latent variables in machine translation were widely used before the
success of deep learning models and consequently NMT. Typically, the
latent variables in these models were of structured nature, such as
representing alignments of phrases in the source and target
sentence~\citep{brown-etal-1993-mathematics}. These structured
models ended up being abandoned in NMT in favour of a direct
decomposition of the model distribution via chain rule without making
any Markov assumptions, which was enabled by advances in
architectures and optimization procedures.

Recently, variational
methods have paved the way for latent variables to be used in NMT,
although usually using a continuous latent variable to represent the
semantic space of the
translation~\citep{zhang2016VariationalNeuralMachine}.
We propose to incorporate structured latent variables into NMT. We
focus on the use of these latent variables as a way to incrementally
guide the model into decomposing translation as a sequence of
subtasks, \eg first encountering a suitable draft translation from
memory and then editing that draft to create the final target
translated sentence, rather than translating from source to target in
one step. Thanks to a probabilistic framework, our methods allow
monolingual data from the source and/or target language, such that it
can be used as semi-supervision of the final NMT model. Our contributions
allow for future NMT models to have:

\begin{itemize}
      \item {\bf Inductive bias}: incorporated into the computational
            graph through the structured latent variables;
      \item {\bf Data efficiency}: thanks to the decomposition of a
            hard task into easier sub-tasks;
      \item {\bf Interpretability}: due to the explanatory power of
            latent variables with a discrete structure.
\end{itemize}

\section{Motivation, Objectives, and Scope}
\label{sec:int_motivation}

Our approach is designed for NMT, but it can be generalized to any
text generation task. Other such tasks include dialogue generation,
language modeling and simultaneous translation.

In Machine Translation, a sentence written in a {\it source
        language}, \eg English, is automatically translated into the chosen
    {\it target language}, \eg Portuguese. In NMT, this translation
process happens through a neural network model, typically a
sequence-to-sequence (Seq2Seq) model that generates the target
sentence conditioned on the source. The most successful models in NMT
use some form of attention
mechanism~\citep{bahdanau2014neural,vaswani2017attention}, and are
described in \secref{sec:nmt}.

In the scope of NMT, of particular interest for this thesis are
language-pairs with limited amount of parallel source-target data
({\it low-resource NMT}), the translation of whole documents from a
source language into a target language ({\it document-level NMT}),
and the ability of a general purpose translation model to learn a
specific domain, such as medical texts ({\it in-domain NMT}). These
particular uses of NMT best showcase the advantage of the latent
variable models proposed, because these are applications that
typically have scarse data and in which current models struggle due
to less capacity to capture the hidden structural nature of the data,
since they lack the inductive bias required to represent that
complexity.

In NMT, we learn a conditional distribution over $\mathcal X \times
    \mathcal Y$, where $\mathcal X$ is the space of source language
sentences and $\mathcal{Y}$ the space of target-language sentences.
This distribution is learned directly by parameterising

\begin{equation}
    Y_j|\theta, x, y_{<j} \sim \Cat(f(x, y_{<j}; \theta))
    \label{eq:nmt_factorize}
\end{equation}

\noindent which takes a variable number of categorical draws in
context auto-regressively, that is, without making Markov
assumptions. On the one hand, the lack of independence assumption is
important as it enables modelling of arbitrarily complex translation
phenomena such as word-order differences and global agreement. On the
other hand, such a rich factorisation demands an abundance of
training resources, in particular, \emph{translation pairs}, not
necessarily available for every language pair of interest.

In order to carve a path towards less data-hungry NMT models, we wish
to incorporate latent structure into these models such that the
translation learning process is guided through these latent
structures. Hopefully, this makes the process easier---instead of
asking the model to map directly from $\mathcal{X}$ to $\mathcal{Y}$,
we introduce latent variables that split that mapping into subtasks.
Particularly, we propose to create a draft translation from the
source sentence, and only then edit that draft translation into a
final sentence. In document-level NMT and in in-domain NMT, we
propose to endow the model with a latent summary of the document or
of the domain, such that the final translation is achieved with
context awareness. Both of these approaches have the potential to
allow learning with less supervised data than in the auto-regressive
factorisation shown in \eqref{eq:nmt_factorize}, thanks to reasonable
assumptions of the generative process. We hypothesize that such models
will allow for better results in language pairs with low-resources,
when compared with vanilla NMT models.

Oftentimes, even though parallel data for some language-pairs is
scarce, monolingual data is available in abundance for many
isolated languages. Having a latent structure that can be inferred
from monolingual data (such as simple sentences, or common phrases)
means that parts of the model can be trained in isolation
only with the monolingual data available, granting the model
with an ability for semi-supervision.

Finally, by using discrete latent structures within the model, we end
up with a window into the translation process. Depending on the
sampled latent variable, the model will output a different
translation. Looking at the possible latent assignments and their
corresponding outputs, we may be able to interpret how the final
translation came to be, to understand the source of model errors,
and to even carefully control details of the final output.

\section{Contributions and Thesis Statement}
\label{sec:int_contributions}

We will now summarize the contributions of this thesis, which will
address the open questions left to answer in the previous section.

\begin{itemize}

      \item \textbf{We use weak supervision by leveraging transfer learning
                  for data efficient sequence-to-sequence models.}
            We show how to leverage a pre-trained large Transformer
            encoder to perform a sequence-to-sequence task on a very
            small dataset. We explore different avenues of parameter
            sharing and initialization in order to make this
            possible.


      \item \textbf{We propose {\boldmath $\alpha$}-\entmaxtext{} with
                  learnable {\boldmath $\alpha$}, a new sparse probability
                  normalization function that learns its own sparsity.}
            We derive the gradient of $\alpha$ in
            $\alpha$-\entmaxtext{}~\citep{entmax}. By letting a
            Transformer learn through this gradient the $\alpha$ of
            each attention head, the attention heads can dynamically
            change their own sparsity during training. This way, each
            attention head can accommodate its sparsity to the role
            it will play in the overall model.

      \item \textbf{We conduct an extensive analysis on the increased
                  transparency of Transformer models that use {\boldmath
                              $\alpha$}-\entmaxtext{} as its normalization function in the
                  attention mechanism.}
            Besides studying the distribution of sparsity
            throughout all attention heads in the Transformer, we also identify
            examples of sharper attention head behavior than what was found in
            previous work, along with the disentanglement of new found behaviors
            thanks to our proposed sparsity.

      \item \textbf{We propose a new method to train discrete and
                  structured latent variable models, based on the
                  efficiency of marginalizing expectations using
                  sparsity.}
            Thanks to this method, we can train these families of
            latent variable models without recurring to any Monte Carlo
            estimation or relaxations of the discreteness into the
            continuous space. In particular, in the unstructured
            case, our method relies on the sparsemax activation
            function. In the structured case, we propose to either
            use SparseMAP~\citep{niculae2018sparsemap} or the
            novel top-$k$ sparsemax.

      \item \textbf{We provide open-source code for each of the
                  methods we have proposed.}
            The respective repositories can be found in each
            of the chapters.


\end{itemize}



\section{Publications}
\label{sec:int_publications}

This thesis is based, in part, on the following publications:

\begin{itemize}

      \item {\bf A Simple and Effective Approach to Automatic
            Post-Editing with Transfer Learning}~\citep{Correia2019}.
            Described in \chapref{chap:ape}, this paper was accepted as a
            poster at ACL 2019.
            
      \item {\bf Unbabel's Submission to the WMT2019 APE Shared Task:
            BERT-based Encoder-Decoder for Automatic
            Post-Editing}~\citep{lopes2019unbabels}. Not included as a specific
            chapter within this thesis, this work describes an Automatic
            Post-Editing model that was submitted to the APE Shared Task at
            WMT2019. This model won the Shared Task, obtaining state-of-the-art
            in APE at the time.
            
      \item {\bf Adaptively Sparse
            Transformers}~\citep{correia2019adaptively}. Described in
            \chapref{chap:adaptsparse}, this paper was accepted for an oral
            presentation at EMNLP 2019.
            
      \item {\bf Efficient Marginalization of Discrete and Structured
            Latent Variables via Sparsity}~\citep{correia2020procneurips}.
            Described in \chapref{chap:sparsemarg}, this work was
            accepted as a spotlight paper at NeurIPS 2020.
            
\end{itemize}
\section{Roadmap}
\label{sec:int_roadmap}

Herein we show the outline of the remainder of this thesis.

We begin in \chapref{chap:background} by reviewing major concepts
that are essential to understand the content of this thesis,
particularly neural network models for NLP, sparse probability
normalization functions, and latent variable models.

In \chapref{chap:ape}, we develop a sequence-to-sequence model for
Automatic Post-Editing using the Transformer architecture and by
harnessing the transfer learning power of a pre-trained large
language model.

In \chapref{chap:adaptsparse}, we augment the Transformer to use
attention modules that allow sparse probabilities that adaptively
change their own sparsity depending on their role within the model.

In \chapref{chap:sparsemarg}, we propose a new training method of
discrete and structured neural latent variable models, which uses
sparse probabilities in order to compute the training objective of
these models, instead of using Monte Carlo methods.

Finally, in \chapref{chap:conclusions}, we summarize the
contributions of the present thesis, address some of the limitations
and open problems of the present work, and discuss exciting future
directions of further research.


\cleardoublepage

\singlespacing