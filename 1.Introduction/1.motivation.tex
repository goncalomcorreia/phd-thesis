\section{Motivation, Objectives, and Scope}
\label{sec:int_motivation}

Our approach is designed for NMT, but it can be generalized to any
text generation task. Other such tasks include dialogue generation,
language modeling and simultaneous translation.

In Machine Translation, a sentence written in a {\it source
        language}, \eg English, is automatically translated into the chosen
    {\it target language}, \eg Portuguese. In NMT, this translation
process happens through a neural network model, typically a
sequence-to-sequence (Seq2Seq) model that generates the target
sentence conditioned on the source. The most successful models in NMT
use some form of attention
mechanism~\citep{bahdanau2014neural,vaswani2017attention}, and are
described in \secref{sec:nmt}.

In the scope of NMT, of particular interest for this thesis are
language-pairs with limited amount of parallel source-target data
({\it low-resource NMT}), the translation of whole documents from a
source language into a target language ({\it document-level NMT}),
and the ability of a general purpose translation model to learn a
specific domain, such as medical texts ({\it in-domain NMT}). These
particular uses of NMT best showcase the advantage of the latent
variable models proposed, because these are applications that
typically have scarse data and in which current models struggle due
to less capacity to capture the hidden structural nature of the data,
since they lack the inductive bias required to represent that
complexity.

In NMT, we learn a conditional distribution over $\mathcal X \times
    \mathcal Y$, where $\mathcal X$ is the space of source language
sentences and $\mathcal{Y}$ the space of target-language sentences.
This distribution is learned directly by parameterising

\begin{equation}
    Y_j|\theta, x, y_{<j} \sim \Cat(f(x, y_{<j}; \theta))
    \label{eq:nmt_factorize}
\end{equation}

\noindent which takes a variable number of categorical draws in
context auto-regressively, that is, without making Markov
assumptions. On the one hand, the lack of independence assumption is
important as it enables modelling of arbitrarily complex translation
phenomena such as word-order differences and global agreement. On the
other hand, such a rich factorisation demands an abundance of
training resources, in particular, \emph{translation pairs}, not
necessarily available for every language pair of interest.

In order to carve a path towards less data-hungry NMT models, we wish
to incorporate latent structure into these models such that the
translation learning process is guided through these latent
structures. Hopefully, this makes the process easier---instead of
asking the model to map directly from $\mathcal{X}$ to $\mathcal{Y}$,
we introduce latent variables that split that mapping into subtasks.
Particularly, we propose to create a draft translation from the
source sentence, and only then edit that draft translation into a
final sentence. In document-level NMT and in in-domain NMT, we
propose to endow the model with a latent summary of the document or
of the domain, such that the final translation is achieved with
context awareness. Both of these approaches have the potential to
allow learning with less supervised data than in the auto-regressive
factorisation shown in \eqref{eq:nmt_factorize}, thanks to reasonable
assumptions of the generative process. We hypothesize that such models
will allow for better results in language pairs with low-resources,
when compared with vanilla NMT models.

Oftentimes, even though parallel data for some language-pairs is
scarce, monolingual data is available in abundance for many
isolated languages. Having a latent structure that can be inferred
from monolingual data (such as simple sentences, or common phrases)
means that parts of the model can be trained in isolation
only with the monolingual data available, granting the model
with an ability for semi-supervision.

Finally, by using discrete latent structures within the model, we end
up with a window into the translation process. Depending on the
sampled latent variable, the model will output a different
translation. Looking at the possible latent assignments and their
corresponding outputs, we may be able to interpret how the final
translation came to be, to understand the source of model errors,
and to even carefully control details of the final output.
