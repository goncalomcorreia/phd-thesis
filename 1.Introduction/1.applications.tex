\section{Aplications}
\label{sec:int_applications}

Our approach is designed for NMT, but it can be generalized to any
text generation task. Other such tasks include dialogue generation,
language modeling and simultaneous translation.

In Machine Translation, a sentence written in a {\it source
language}, \eg English, is automatically translated into the chosen
{\it target language}, \eg Portuguese. In NMT, this translation
process happens through a neural network model, typically a
sequence-to-sequence (Seq2Seq) model that generates the target
sentence conditioned on the source. The most successful models in NMT
use some form of attention
mechanism~\citep{bahdanau2014neural,vaswani2017attention}, and are
described in \secref{sec:nmt}.

In the scope of NMT, of particular interest for this thesis are
language-pairs with limited amount of parallel source-target data
({\it low-resource NMT}), the translation of whole documents from a
source language into a target language ({\it document-level NMT}),
and the ability of a general purpose translation model to learn a
specific domain, such as medical texts ({\it in-domain NMT}). These
particular uses of NMT best showcase the advantage of the latent
variable models proposed, because these are applications that
typically have scarse data and in which current models struggle due
to less capacity to capture the hidden structural nature of the data,
since they lack the inductive bias required to represent that
complexity.